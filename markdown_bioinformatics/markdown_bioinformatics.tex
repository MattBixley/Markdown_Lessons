\documentclass{bioinfo}
\copyrightyear{2016} \pubyear{2016}

%% Some pieces required from the pandoc template
\providecommand{\tightlist}{%
  \setlength{\itemsep}{0pt}\setlength{\parskip}{0pt}}


% Pandoc citation processing


% hyperref makes the margins screwy.
% https://groups.google.com/forum/#!topic/latexusersgroup/4W_SwGk6zx4
% http://ansuz.sooke.bc.ca/software/latex-tricks.php
% \usepackage[colorlinks=true, allcolors=blue]{hyperref}

\access{Advance Access Publication Date:   }
\appnotes{Manuscript Category}

\begin{document}
\firstpage{1}

\subtitle{Application Note}

\title[Bioinformatics Rmd Template]{Template for preparing a submission to Bioinformatics using RMarkdown}

\author[FirstAuthorLastName \textit{et~al}.]{
Alice Anonymous\,\textsuperscript{1,2},
Bob Security\,\textsuperscript{1},
}

\address{
\textsuperscript{1}Some Institute of Technology, Department, Street, City, State, Zip\\
\textsuperscript{2}Another University Department, Street, City, State, Zip\\
}

\corresp{To whom correspondence should be addressed. E-mail: bob@email.com}

\history{Received on XXX; revised on XXX; accepted on XXX}

\editor{Associate Editor: XXX}

\abstract{
\textbf{Motivation:} This section should specifically state the scientific question within
the context of the field of study.\\
\textbf{Results:} This section should summarize the scientific advance or novel results of
the study, and its impact on computational biology.\\
\textbf{Availability:} This section should state software availability if the paper focuses
mainly on software development or on the implementation of an algorithm.
Examples are: `Freely available on the web at XXX.' Website implemented
in Perl, MySQL and Apache, with all major browsers supported'; or
`Source code and binaries freely available for download at URL,
implemented in C++ and supported on linux and MS Windows'. The complete
address (URL) should be given. If the manuscript describes new software
tools or the implementation of novel algorithms the software must be
freely available to non-commercial users. Authors must also ensure that
the software is available for a full TWO YEARS following publication.
The editors of Bioinformatics encourage authors to make their source
code available and, if possible, to provide access through an open
source license.\\
\textbf{Contact:}bob@email.com\\
\textbf{Supplementary information:} Supplementary data are available at Bioinformatics Online.}

\maketitle

\section{Introduction}

Cite others using bracket notation \citep{pepe2003statistical}. Can also
cite with \citet{zou2005regularization}.

Instructions for authors are available
\href{http://www.oxfordjournals.org/our_journals/bioinformatics/for_authors/general.html}{online}.

Introduce your topic. Lorem ipsum ad nauseum. Introduce your topic.
Lorem ipsum ad nauseum. Introduce your topic. Lorem ipsum ad nauseum.
Introduce your topic. Lorem ipsum ad nauseum. Introduce your topic.
Lorem ipsum ad nauseum.

Introduce your topic. Lorem ipsum ad nauseum. Introduce your topic.
Lorem ipsum ad nauseum. Introduce your topic. Lorem ipsum ad nauseum.
Introduce your topic. Lorem ipsum ad nauseum. Introduce your topic.
Lorem ipsum ad nauseum.

\section{Approach}

Here is how to include math equations in the document (bounded by
\texttt{\$\$}):

\[
\begin{aligned}
(x+y)^3&=(x+y)(x+y)^2\\
       &=(x+y)(x^2+2xy+y^2) \label{eqn:example} \\
       &=x^3+3x^2y+3xy^3+x^3. 
\end{aligned}
\]

Describe the approach. Lorem ipsum ad nauseum. Introduce your topic.
Lorem ipsum ad nauseum. Introduce your topic. Lorem ipsum ad nauseum.
Introduce your topic. Lorem ipsum ad nauseum. Introduce your topic.
Lorem ipsum ad nauseum.

Describe the approach. Lorem ipsum ad nauseum. Introduce your topic.
Lorem ipsum ad nauseum. Introduce your topic. Lorem ipsum ad nauseum.
Introduce your topic. Lorem ipsum ad nauseum. Introduce your topic.
Lorem ipsum ad nauseum.

Describe the approach. Lorem ipsum ad nauseum. Introduce your topic.
Lorem ipsum ad nauseum. Introduce your topic. Lorem ipsum ad nauseum.
Introduce your topic. Lorem ipsum ad nauseum. Introduce your topic.
Lorem ipsum ad nauseum.

Describe the approach. Lorem ipsum ad nauseum. Introduce your topic.
Lorem ipsum ad nauseum. Introduce your topic. Lorem ipsum ad nauseum.
Introduce your topic. Lorem ipsum ad nauseum. Introduce your topic.
Lorem ipsum ad nauseum.

\begin{figure}
\centering
\includegraphics{markdown_bioinformatics_files/figure-latex/figure-1.pdf}
\caption{Figure from an Rmd chunk.}
\end{figure}

\section{Methods}

Detailed methods. Lorem ipsum ad nauseum. Introduce your topic. Lorem
ipsum ad nauseum. Introduce your topic. Lorem ipsum ad nauseum.
Introduce your topic. Lorem ipsum ad nauseum. Introduce your topic.
Lorem ipsum ad nauseum.

Detailed methods. Lorem ipsum ad nauseum. Introduce your topic. Lorem
ipsum ad nauseum. Introduce your topic. Lorem ipsum ad nauseum.
Introduce your topic. Lorem ipsum ad nauseum. Introduce your topic.
Lorem ipsum ad nauseum.

Detailed methods. Lorem ipsum ad nauseum. Introduce your topic. Lorem
ipsum ad nauseum. Introduce your topic. Lorem ipsum ad nauseum.
Introduce your topic. Lorem ipsum ad nauseum. Introduce your topic.
Lorem ipsum ad nauseum.

\subsection{Sub-Method}

Details for Method 1. Lorem ipsum ad nauseum. Introduce your topic.
Lorem ipsum ad nauseum. Introduce your topic. Lorem ipsum ad nauseum.
Introduce your topic. Lorem ipsum ad nauseum. Introduce your topic.
Lorem ipsum ad nauseum.

Details for Method 1. Lorem ipsum ad nauseum. Introduce your topic.
Lorem ipsum ad nauseum. Introduce your topic. Lorem ipsum ad nauseum.
Introduce your topic. Lorem ipsum ad nauseum. Introduce your topic.
Lorem ipsum ad nauseum.

Details for Method 1. Lorem ipsum ad nauseum. Introduce your topic.
Lorem ipsum ad nauseum. Introduce your topic. Lorem ipsum ad nauseum.
Introduce your topic. Lorem ipsum ad nauseum. Introduce your topic.
Lorem ipsum ad nauseum.

Details for Method 1. Lorem ipsum ad nauseum. Introduce your topic.
Lorem ipsum ad nauseum. Introduce your topic. Lorem ipsum ad nauseum.
Introduce your topic. Lorem ipsum ad nauseum. Introduce your topic.
Lorem ipsum ad nauseum.

\subsection{Method 2}

Details for Method 2. Lorem ipsum ad nauseum. Introduce your topic.
Lorem ipsum ad nauseum. Introduce your topic. Lorem ipsum ad nauseum.
Introduce your topic. Lorem ipsum ad nauseum. Introduce your topic.
Lorem ipsum ad nauseum.

Details for Method 2. Lorem ipsum ad nauseum. Introduce your topic.
Lorem ipsum ad nauseum. Introduce your topic. Lorem ipsum ad nauseum.
Introduce your topic. Lorem ipsum ad nauseum. Introduce your topic.
Lorem ipsum ad nauseum.

Details for Method 2. Lorem ipsum ad nauseum. Introduce your topic.
Lorem ipsum ad nauseum. Introduce your topic. Lorem ipsum ad nauseum.
Introduce your topic. Lorem ipsum ad nauseum. Introduce your topic.
Lorem ipsum ad nauseum.

Details for Method 2. Lorem ipsum ad nauseum. Introduce your topic.
Lorem ipsum ad nauseum. Introduce your topic. Lorem ipsum ad nauseum.
Introduce your topic. Lorem ipsum ad nauseum. Introduce your topic.
Lorem ipsum ad nauseum.

\section{Discussion}

Discussion of results. Lorem ipsum ad nauseum. Introduce your topic.
Lorem ipsum ad nauseum. Introduce your topic. Lorem ipsum ad nauseum.
Introduce your topic. Lorem ipsum ad nauseum. Introduce your topic.
Lorem ipsum ad nauseum.

Discussion of results. Lorem ipsum ad nauseum. Introduce your topic.
Lorem ipsum ad nauseum. Introduce your topic. Lorem ipsum ad nauseum.
Introduce your topic. Lorem ipsum ad nauseum. Introduce your topic.
Lorem ipsum ad nauseum.

Discussion of results. Lorem ipsum ad nauseum. Introduce your topic.
Lorem ipsum ad nauseum. Introduce your topic. Lorem ipsum ad nauseum.
Introduce your topic. Lorem ipsum ad nauseum. Introduce your topic.
Lorem ipsum ad nauseum.

Discussion of results. Lorem ipsum ad nauseum. Introduce your topic.
Lorem ipsum ad nauseum. Introduce your topic. Lorem ipsum ad nauseum.
Introduce your topic. Lorem ipsum ad nauseum. Introduce your topic.
Lorem ipsum ad nauseum.

\section{Conclusion}

Anything else? Lorem ipsum ad nauseum. Introduce your topic. Lorem ipsum
ad nauseum. Introduce your topic. Lorem ipsum ad nauseum. Introduce your
topic. Lorem ipsum ad nauseum. Introduce your topic. Lorem ipsum ad
nauseum.

Anything else? Lorem ipsum ad nauseum. Introduce your topic. Lorem ipsum
ad nauseum. Introduce your topic. Lorem ipsum ad nauseum. Introduce your
topic. Lorem ipsum ad nauseum. Introduce your topic. Lorem ipsum ad
nauseum.

Anything else? Lorem ipsum ad nauseum. Introduce your topic. Lorem ipsum
ad nauseum. Introduce your topic. Lorem ipsum ad nauseum. Introduce your
topic. Lorem ipsum ad nauseum. Introduce your topic. Lorem ipsum ad
nauseum.

Anything else? Lorem ipsum ad nauseum. Introduce your topic. Lorem ipsum
ad nauseum. Introduce your topic. Lorem ipsum ad nauseum. Introduce your
topic. Lorem ipsum ad nauseum. Introduce your topic. Lorem ipsum ad
nauseum.

\section*{Acknowledgements}
\addcontentsline{toc}{section}{Acknowledgements}

These should be included at the end of the text and not in footnotes.
Please ensure you acknowledge all sources of funding, see funding
section below.

Details of all funding sources for the work in question should be given
in a separate section entitled `Funding'. This should appear before the
`Acknowledgements' section.

\section*{Funding}
\addcontentsline{toc}{section}{Funding}

The following rules should be followed:

\begin{itemize}
\tightlist
\item
  The sentence should begin: `This work was supported by \ldots{}' -
\item
  The full official funding agency name should be given, i.e.~`National
  Institutes of Health', not `NIH' (full RIN-approved list of UK funding
  agencies)
\item
  Grant numbers should be given in brackets as follows: `{[}grant number
  xxxx{]}'
\item
  Multiple grant numbers should be separated by a comma as follows:
  `{[}grant numbers xxxx, yyyy{]}'
\item
  Agencies should be separated by a semi-colon (plus `and' before the
  last funding agency)
\item
  Where individuals need to be specified for certain sources of funding
  the following text should be added after the relevant agency or grant
  number `to {[}author initials{]}'.
\end{itemize}

An example is given here: `This work was supported by the National
Institutes of Health {[}AA123456 to C.S., BB765432 to M.H.{]}; and the
Alcohol \& Education Research Council {[}hfygr667789{]}.'

Oxford Journals will deposit all NIH-funded articles in PubMed Central.
See Depositing articles in repositories -- information for authors for
details. Authors must ensure that manuscripts are clearly indicated as
NIH-funded using the guidelines above.


% Bibliography
\bibliographystyle{natbib}
\bibliography{bibliography.bib}

\end{document}
